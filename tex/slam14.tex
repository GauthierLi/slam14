% !TEX program=xelatex
\documentclass{article}

\usepackage{ctex}
\usepackage{minted}
\usepackage{xcolor}
\usepackage{hyperref}
\usepackage{setspace}
% 表格
\usepackage{booktabs, diagbox, multirow}
\usepackage{amsmath, amssymb, amsfonts}
\usepackage{listings, footnote, enumerate, enumitem,
            fontspec, geometry, graphicx}

\definecolor{lightgray}{RGB}{220,220,220}
\definecolor{NavyBlue}{RGB}{0,0,128}
% 设置行间距
\setstretch{1.25}
\geometry{a4paper,left=2cm,right=2cm,top=2.5cm,bottom=2cm}
% enumerate 设置行间距
\setlist[enumerate]{itemsep=2pt, topsep=3pt, parsep=0pt, leftmargin=2cm}
\setminted{
      linenos, % 启用行号
      numbersep=5pt, % 行号与代码间的距离
      frame=single, % 单一边框
      fontsize=\footnotesize, % 字体大小
      framesep=2mm, % 边框与代码间的距离
    }
\renewcommand{\arraystretch}{1.5}

\title{\textbf{slam14 笔记}}
\author{Gauthierli\footnote{email: lwklxh@163.com}}
\date{\today}

\begin{document}
\maketitle
\newpage
\tableofcontents
\newpage

\section{CmakeList使用指南}
\subsection{基础使用步骤}
CMakeLists.txt 是一个文本文件,用于定义如何使用 CMake 工具构建(编译)你的软件项目。
CMake 是一个跨平台的开源系统,用来管理软件构建过程。它简化了在不同操作系统上构建项目的过程,支持多种编译工具如GNU make, Apple Xcode, MSBuild等。

使用 CMakeLists.txt 的基本步骤:
\begin{enumerate}[leftmargin=1cm]
  \item 创建 CMakeLists.txt 文件:在你的项目的根目录下创建这个文件,并根据需要填写内容。
  \item 设定最低版本要求:指定你的项目所需的最低CMake版本。例如:
\begin{minted}{bash}
  cmake_minimum_required(VERSION 3.10)
\end{minted}
  \item 定义项目名称和版本号:使用 project 命令来定义项目的名称以及可选的版本号、语言和支持的库。例如:
\begin{minted}{bash}
project(MyProject VERSION 1.0 LANGUAGES CXX)
\end{minted}
  \item 添加源文件或子目录:你可以直接添加源文件或者包含其他含有 CMakeLists.txt 的子目录。
\begin{minted}{bash}
add_executable(MyExecutable main.cpp) # 添加源文件
add_subdirectory(src) # 包含子目录
\end{minted}
  \item 查找并链接库:如果你的项目依赖于外部库,可以使用 find\_package 来定位这些库,
    并使用 target\_link\_libraries 将它们与目标文件链接起来。
\begin{minted}{bash}
find_package(OpenCV REQUIRED)
include_directories(${OpenCV_INCLUDE_DIRS})
add_executable(use_opencv use_opencv.cpp )
target_link_libraries(use_opencv ${OpenCV_LIBS} absl::str_format glog::glog gflags)
\end{minted}
\item 可以设置关键字来指定生成可执行文件/库文件的路径\footnote{
PROJECT\_SOURCE\_DIR 在整个项目的所有 CMakeLists.txt 文件和脚本中都是相同的值,
因为它指向的是最初调用 project() 的那个 CMakeLists.txt 文件所在的目录。}
\begin{minted}{bash}
set(EXECUTABLE_OUTPUT_PATH ${CMAKE_BINARY_DIR}/bin) # 编译路径下的bin文件夹中
set(LIBRARY_OUTPUT_PATH ${PROJECT_SOURCE_DIR}/libs) # 项目路径下的libs文件夹中
\end{minted}
\item 生成构建文件:从命令行进入你的项目目录(即包含 CMakeLists.txt 的目录),
  运行以下命令生成适合你的开发环境的构建文件:
\begin{minted}{bash}
mkdir build && cd build
cmake ..
\end{minted}
\end{enumerate}

C++项目中构建结束后会生成一系列的make文件,然后可以使用 
make 命令来编译项目并生成可执行文件以及库文件:
\begin{minted}{bash}
make -j$(nproc)
\end{minted}

\subsection{CmakeLists进阶使用方法}
CMakeLists.txt文件中,使用最多的是命令\footnote{在CMakeLists.txt中,命令名不区分大小写,
可以使用大小写字母书写命令名},譬如上例中的 cmake\_minimum\_required 
project 都 是命令;命令的使用方式有点类似于 C 语言中的函数,因为命令后面需要提供一对括号,
并且通常需要我们提供参数,多个参数使用空格分隔而不是逗号“,”,这是与函数不同的地方。

其中常用的命令如下:
\begin{enumerate}[leftmargin=1cm]

\item set 命令常用于设置变量,例如:
\begin{minted}{bash}
# 设置变量 MY_VAL
set(MY_VAL "Hello World!") 
\end{minted}
我们也可以用一个变量来表示多个源文件

\begin{minted}{bash}
set(SRC_LIST a.cpp b.cpp c.cpp) 
add_executable(${PROJECT_NAME} ${SRC_LIST})
\end{minted}

\item 在CMakeLists.txt 中添加如下这句可以打印出时间辍
\begin{minted}{bash}
  string(TIMESTAMP COMPILE_TIME %Y%m%d-%H%M%S)
\end{minted}

\item 常用的宏变量,可以使用set指定,或者在编译的时候使用-D指定

\begin{table}[!h]
  \caption{常见宏定义表}
  \label{hong}
  \centering 
  \begin{tabular}{|c|c|}
    \hline
    \textbf{宏定义} & \textbf{说明} \\ \hline
    CMAKE\_SOURCE\_DIR & 顶层源代码目录 \\ \hline
    CMAKE\_BINARY\_DIR & 顶层二进制目录 \\ \hline
    PROJECT\_SOURCE\_DIR & 当前项目的源代码目录 \\ \hline
    PROJECT\_BINARY\_DIR & 当前项目的二进制目录 \\ \hline
    CMAKE\_CURRENT\_SOURCE\_DIR & 当前正在处理的 CMakeLists.txt 所在的目录 \\ \hline 
    CMAKE\_CURRENT\_BINARY\_DIR & 当前正在处理的 CMakeLists.txt 对应的构建目录 \\ \hline 
    CMAKE\_C\_FLAGS & 用于设置 C 编译器的选项 \\ \hline 
    CMAKE\_CXX\_FLAGS & 设置 C++ 编译器的选项 \\ \hline 
    CMAKE\_BUILD\_TYPE & 能指定构建类型,像Debug、Release、RelWithDebInfo等 \\ \hline
    CMAKE\_C\_COMPILER & 表示 C 编译器的路径 \\ \hline 
    CMAKE\_CXX\_COMPILER & 为 C++ 编译器的路径 \\ \hline 
    PROJECT\_NAME & 当前项目的名称,通过project()命令进行设置 \\ \hline 
    PROJECT\_VERSION & 当前项目的版本号,同样由project()命令设置 \\ \hline 
    CMAKE\_PROJECT\_NAME & 顶层项目的名称 \\ \hline 
    CMAKE\_INSTALL\_PREFIX & 安装路径的前缀,默认为/usr/local \\ \hline
    CMAKE\_INSTALL\_BINDIR & 可执行文件的安装目录,默认是bin \\ \hline
    CMAKE\_INSTALL\_LIBDIR & 库文件的安装目录,默认是lib \\ \hline
    CMAKE\_INSTALL\_INCLUDEDIR & 头文件的安装目录,默认是include \\ \hline
    CMAKE\_EXPORT\_COMPILE\_COMMANDS & 生成compile\_commands.json文件,供 IDE 使用 \\ \hline
    
  \end{tabular}
\end{table}

  \item message 常用于打印调试信息,类似于 printf 函数。
  \item 判断语句的使用如下,使用方法如下:
  \begin{minted}{bash}
    if(WIN32)  # Windows系统
      message("Building on Windows")
      set(PLATFORM_LIBS ws2_32)  # Windows网络库
    elseif(APPLE)  # macOS系统
      message("Building on macOS")
      set(PLATFORM_LIBS "-framework Cocoa")  # macOS Cocoa框架
    elseif(UNIX)  # Linux或其他UNIX系统
      message("Building on Linux or UNIX")
      set(PLATFORM_LIBS pthread)  # Linux线程库
    endif()

  # 判断变量是否定义
  if(DEFINED MY_VARIABLE)
    message("MY_VARIABLE is defined")
  endif()

  # 判断变量是否为空
  if(MY_VARIABLE)  # 变量存在且非空、非0、非OFF、非FALSE等
    message("MY_VARIABLE is not empty")
  else()
    message("MY_VARIABLE is empty or false")
  endif()

  # 判断变量是否等于某个值
  if(MY_VARIABLE STREQUAL "value")
    message("MY_VARIABLE equals 'value'")
  endif()

  \end{minted}

\item 自定义选项可以使用option与条件判断结合
  \begin{minted}{bash}
  option(ENABLE_TESTING "Build tests" ON)
  if(ENABLE_TESTING)
    enable_testing()
    add_subdirectory(tests)
  endif()
  \end{minted}

\end{enumerate}

\newpage
\section{李群和李代数之间的理解}
\subsection{$\mathfrak{so}(3)$ 上元素$\omega$的意义}
假设拿到一个旋转矩阵 $ R \in SO(3)$ ,即三维空间中的合法的旋转(满足:正交矩阵+行列式为1)。那么
如何求出他在李代数$\mathfrak{so}(3)$上对应的元素,记作$\omega ^{\wedge} $,
那么这个$\omega ^{\wedge} $的意义是什么?


\textbf{数学表达}:
对于任意 $ R \in \text{SO}(3) $,可以表示为:

$$
R = \exp(\omega^\wedge)
$$

其中:
\begin{enumerate}
  \item $ \omega \in \mathbb{R}^3 $ 是一个三维向量,表示绕某个轴旋转的角度 × 方向;
  \item $ \omega^\wedge \in \mathfrak{so}(3) $ 是一个 3×3 的反对称矩阵(即李代数元素);
  \item $\exp: \mathfrak{so}(3) \to \text{SO}(3)$ 是指数映射。
\end{enumerate}


几何意义解释:

\textbf{1. $ \omega $ 表示旋转的轴角表示}
\begin{enumerate}
  \item 向量 $ \omega \in \mathbb{R}^3 $ 的方向就是旋转轴;
  \item 它的模长 $ |\omega| $ 表示绕该轴旋转的角度(单位通常是弧度);
  \item 所以 $ \omega $ 就是所谓的 \textbf{轴角表示(Axis-Angle Representation)}。
\end{enumerate}

eg:如果 $ \omega = (0, 0, \theta) $,表示绕 z 轴旋转 $ \theta $ 弧度。

\textbf{2. $\omega^\wedge$ 是旋转的“无穷小生成元”}
\begin{enumerate}
\item 李代数 $ \mathfrak{so}(3) $ 中的元素 $ \omega^\wedge $ 可以看作是旋转的“速度”或“微分生成器”;
\item 如果你把 $ R(t) = \exp(t\omega^\wedge) $ 看作是随着时间 t 变化的连续旋转,那么:
  \begin{enumerate}
    \item 在 $ t=0 $ 处,导数是 $ \left.\frac{d}{dt}\right|_{t=0} R(t) = \omega^\wedge $
    \item 所以 $ \omega^\wedge $ 表示的是这个旋转运动在初始时刻的“角速度”。
  \end{enumerate}
\end{enumerate}

\textbf{3. 便于做优化、插值和扰动分析}
\begin{enumerate}
\item 在机器人、SLAM、计算机视觉等领域中,经常需要对旋转进行优化或插值;
\item 直接在 SO(3) 上操作不方便(比如不能直接加),
  但可以在李代数 $ \mathfrak{so}(3) \cong \mathbb{R}^3 $ 上进行线性运算;

 eg: 给某个旋转加上一个小扰动:
  $$
  R' = R \cdot \exp(\delta^\wedge)
  $$
  其中 $ \delta \in \mathbb{R}^3 $ 是一个小的扰动向量。

\end{enumerate}

\textbf{总结}:
 给定一个旋转矩阵 $ R \in \text{SO}(3) $,它在李代数 $ \mathfrak{so}(3) $
 上的对应元素 $ \omega^\wedge $(或其向量形式 $ \omega $)
 表示了这个旋转的\textbf{旋转轴和角度信息},
 同时也代表了从单位旋转出发通过指数映射到达当前旋转所需的\textbf{旋转速度/方向}。

\end{document}
