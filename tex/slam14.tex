% !TEX program=xelatex
\documentclass{article}

\usepackage{ctex}
\usepackage{xcolor}
\usepackage{hyperref}
\usepackage{setspace}
\usepackage{amsmath, amssymb, amsfonts}
\usepackage{listings, footnote, enumerate, enumitem,
            fontspec, geometry, graphicx}

\definecolor{lightgray}{RGB}{220,220,220}
\definecolor{NavyBlue}{RGB}{0,0,128}
% 设置行间距
\setstretch{1.25}
\geometry{a4paper,left=2cm,right=2cm,top=2.5cm,bottom=2cm}
% enumerate 设置行间距
\setlist[enumerate]{itemsep=2pt, topsep=3pt, parsep=0pt, leftmargin=2cm}

\title{\textbf{slam14 笔记}}
\author{Gauthierli\footnote{email: lwklxh@163.com}}
\date{\today}

\begin{document}
\maketitle
\newpage
\tableofcontents
\newpage

\section{李群和李代数之间的理解}
\subsection{$\mathfrak{so}(3)$ 上元素$\omega$的意义}
假设拿到一个旋转矩阵 $ R \in SO(3)$ ,即三维空间中的合法的旋转(满足:正交矩阵+行列式为1)。那么
如何求出他在李代数$\mathfrak{so}(3)$上对应的元素,记作$\omega ^{\wedge} $,
那么这个$\omega ^{\wedge} $的意义是什么?


\textbf{数学表达}:
对于任意 $ R \in \text{SO}(3) $,可以表示为:

$$
R = \exp(\omega^\wedge)
$$

其中:
\begin{enumerate}
  \item $ \omega \in \mathbb{R}^3 $ 是一个三维向量,表示绕某个轴旋转的角度 × 方向;
  \item $ \omega^\wedge \in \mathfrak{so}(3) $ 是一个 3×3 的反对称矩阵(即李代数元素);
  \item $\exp: \mathfrak{so}(3) \to \text{SO}(3)$ 是指数映射。
\end{enumerate}


几何意义解释:

\textbf{1. $ \omega $ 表示旋转的轴角表示}
\begin{enumerate}
  \item 向量 $ \omega \in \mathbb{R}^3 $ 的方向就是旋转轴;
  \item 它的模长 $ |\omega| $ 表示绕该轴旋转的角度(单位通常是弧度);
  \item 所以 $ \omega $ 就是所谓的 \textbf{轴角表示(Axis-Angle Representation)}。
\end{enumerate}

eg:如果 $ \omega = (0, 0, \theta) $,表示绕 z 轴旋转 $ \theta $ 弧度。

\textbf{2. $\omega^\wedge$ 是旋转的“无穷小生成元”}
\begin{enumerate}
\item 李代数 $ \mathfrak{so}(3) $ 中的元素 $ \omega^\wedge $ 可以看作是旋转的“速度”或“微分生成器”;
\item 如果你把 $ R(t) = \exp(t\omega^\wedge) $ 看作是随着时间 t 变化的连续旋转,那么:
  \begin{enumerate}
    \item 在 $ t=0 $ 处,导数是 $ \left.\frac{d}{dt}\right|_{t=0} R(t) = \omega^\wedge $
    \item 所以 $ \omega^\wedge $ 表示的是这个旋转运动在初始时刻的“角速度”。
  \end{enumerate}
\end{enumerate}

\textbf{3. 便于做优化、插值和扰动分析}
\begin{enumerate}
\item 在机器人、SLAM、计算机视觉等领域中,经常需要对旋转进行优化或插值;
\item 直接在 SO(3) 上操作不方便(比如不能直接加),
  但可以在李代数 $ \mathfrak{so}(3) \cong \mathbb{R}^3 $ 上进行线性运算;

 eg: 给某个旋转加上一个小扰动:
  $$
  R' = R \cdot \exp(\delta^\wedge)
  $$
  其中 $ \delta \in \mathbb{R}^3 $ 是一个小的扰动向量。

\end{enumerate}

\textbf{总结}:
 给定一个旋转矩阵 $ R \in \text{SO}(3) $,它在李代数 $ \mathfrak{so}(3) $
 上的对应元素 $ \omega^\wedge $(或其向量形式 $ \omega $)
 表示了这个旋转的\textbf{旋转轴和角度信息},
 同时也代表了从单位旋转出发通过指数映射到达当前旋转所需的\textbf{旋转速度/方向}。

\end{document}
